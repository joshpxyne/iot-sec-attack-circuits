\section{Related Work}
\label{sec:related_work}

In computer security, a \textit{vulnerability} is defined as a weakness of a system that can be exploited by an attacker. The attacker may then perform unauthorized activities within the system. Alternatively, an \textit{exposure} is a software error in the system that allows the attacker to gain access to system data and conduct information gathering activities. The attacker may accompany this by hiding unauthorized system activity from associated monitoring services. Subsequently, the Common Vulnerabilities and Exposures (CVE) system \cite{mell2006common} is a built reference for publicly identified information-security vulnerabilities and exposures. The system is maintained by the Mitre Corporation\footnote{mitre.org}. CVE entries are primarily composed of identifiers, descriptions, references and the date at which the CVE entry was created. The Mitre Corporation also maintains the Common Weakness Enumeration (CWE) system \cite{martin2007common}, which categorizes software weaknesses and vulnerabilities. The combined use to CVE and CWE allows organizations to select appropriate software tools for internal usage. Our work uses the CVE and CWE systems as a standardized source of information to generate a representation of possible attacks. Additionally, the Common Vulnerability Scoring System (CVSS) \cite{mell2006common} provides a way to capture the principal characteristics of a vulnerability and produce a numerical score reflecting its severity. We build on these existing scores to evaluate device and network vulnerability.

Work in \cite{acar2018peek} discusses the threat to user privacy in a smart-home environment, using multi-stage privacy attacks. The attacks are evaluated to be effective in determining the state and actions of the devices in both unencrypted and encrypted communication settings. The work is evaluated using common IoT devices and multiple attack protocols (passive, observer attack). With the use of a large degree of automation in activity detection and identification (using machine learning techniques and network traffic data mining), this work exemplifies the security threats faced by smart-home networks. Work in \cite{apthorpe2017spying} also emphasizes the security risks of smart-home networks using commercially available smart-home devices with encrypted communication. The authors explore attacks by first identifying the device (using Domain Name System (DNS) queries or device fingerprinting) and then inference of activities based on changes in network traffic. Network traffic-based threats are also highlighted in \cite{apthorpe2017smart}, where the authors demonstrate an attacker that passively observes encrypted network traffic to infer sensitive details about network users.

Work in \cite{miettinen2017iot} examines the security flaws for smart-home networks with specific interest in the exploitation of the lack of mechanisms for firmware updates or patches for security vulnerabilities. The authors propose a system to identify the types of devices that are connected and suggest the use of appropriate communication constraints, given that knowledge. The device type for their work is the enumeration of a specific device. Work in \cite{miettinen2017iot} is, however, limited to the formulation of a method to identify a given device, and does not provide a metric to determine how vulnerable a device is. Network attacks are modeled using attack graphs in \cite{ingols2009modeling}. The authors propose a scalable model zero-day exploits \cite{turner2005symantec} and client-side attacks \cite{choo2011cyber,chang2009your}, in contrast to prior attack graph systems that focus on server-side vulnerabilities \cite{noel2008optimal,lippmann2005annotated}. The work, however, focuses on modeling such attacks and corresponding countermeasures, but does not provide a means to quantify the relative impacts of different attacks. Work in \cite{matsuoka2015security} discusses the provision of security objectives for smart-home networks. While it does not discuss the underlying detection mechanism used, the application of security flaw detection is aligned with the motivation for our work.

Term Frequency-Inverse Document Frequency (TF-IDF) \cite{leskovec2014mining} is a numerical statistic used in information retrieval to determine the relative importance of words in a document. TF-IDF and TF and IDF individually may also serve as heuristics for weighting words. Our work leverages TF-IDF to compute attack meta-data for a given CVE entry. TextRank \cite{mihalcea2004textrank,PyTextRank} is a graph-based ranking model for text processing. It is inspired by recursive graph-based ranking algorithms such as HITS \cite{kleinberg1999authoritative} and PageRank \cite{page1999pagerank}, using a voting mechanism. While TextRank may be used for sentence extraction and text summarization, our work uses the information stored in the intermediate process: the extraction and ranking of phrases.